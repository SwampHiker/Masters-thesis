\newpage
\begin{center}
	\textbf{\large ГЛАВА 1}

	\textbf{\large ОПИСАНИЕ БЕЛКОВЫХ МОЛЕКУЛ И КОМПЛЕКСОВ И СВЯЗАННЫХ ЗАДАЧ}
\end{center}
\refstepcounter{chapter}
\addcontentsline{toc}{chapter}{1. ОПИСАНИЕ БЕЛКОВЫХ МОЛЕКУЛ И КОМПЛЕКСОВ И СВЯЗАННЫХ ЗАДАЧ}

%copypaste а там поглядим

Полипептиды – биополимерные органические соединения, являющиеся основой для всей, известной на данный момент, жизни. Они состоят из аминокислотных остатков, связанных пептидной связью. Полипептиды условно разделяют на пептиды (состоят из менее чем 50 остатков) и белки (более чем 50 остатков). Далее, будем говорить в основном о белках. Части аминокислотных молекул, участвующие в пептидной связи, образуют главную цепь белка. К главной цепи белка присоединены многочисленные боковые цепи – части аминокислотных остатков, не участвующие в пептидной связи, но определяющие функциональную роль соответствующих участков белка. Существует 20 основных аминокислот, встречающихся повсеместно. Кроме 20 основных, существуют другие, более редкие аминокислоты, однако на практике, ими всегда пренебрегают. При изучении белков, выделяют 4 уровня структуры:
\begin{enumerate}
\item Первичный уровень – это линейная последовательность аминокислотных остатков. Эта последовательность устанавливается с секвенированием. С одной стороны, этого уровня, самого по себе, не достаточно для решения большинства задач обработки белковых соединений. Однако, данный уровень кодирует все более высокие структурные уровни. Кроме того, первичный уровень позволяет устанавливать эволюционную близость между двумя белковыми соединениями.
\item Вторичный уровень – последовательные участки главной цепи белковой молекулы склонны образовывать регулярные структуры: β-листы, α-спирали; нерегулярные структуры: изгибы и петли; а также неупорядоченные участки (которые можно считать нерегулярной последовательностью очень коротких регулярных участков).
\item Третичный уровень – полная трёхмерная структура молекулы. Включает в себя структуры вторичного уровня, их взаимное расположение, а также положение всех боковых цепей. Представляет собой основной интерес, так как позволяет строить фармакофорную модель, проводить докинг с малыми молекулами, проводить анализ молекулярной динамики и т.д. Трёхмерную структуру полипептидной модели называют её свёрткой. 
\item Четвертичный уровень – существует для белковых молекул, образующих комплексы – соединения, образованные слабыми связями. Представляет собой третичные структуры всех компонент и их взаимное расположение. Часто встречаются комплексы, состоящие из нескольких одинаковых компонент. В зависимости от числа таких компонент, выделяют различные виды симметрий \cite{symmetry}. Важным отличием белков от пептидов в контексте четвертичного уровня является то, что большие белковые молекулы почти не изменяются структурно при взаимодействии друг с другом. В то же время пептид, при взаимодействии с большой белковой молекулой, может значительно поменять собственную свёртку.
Важными фундаментальными задачами являются предсказание сворачивания белка, предсказание взаимодействия белков и дизайн белков. Первая представляет собой предсказание третичной структуры по первичной; вторая – предсказание четвертичной структуры по третичной; третья – поиск первичной структуры, удовлетворяющей заданной третичной. Рассмотрим существующие методы и подходы к решению задачи предсказания взаимодействия.
\end{enumerate}
\section{Подходы к предсказанию взаимодействия}
Выделим несколько групп методов, используемых для предсказания взаимодействия. Первой такой группой будут методы свободного докинга. В них могут использоваться эмпирические поля, скоринговые функции (в том числе, полученные с помощью машинного обучения). Эти поля и функции соответствуют свободной энергии комплекса. Минимум энергии должен соответствовать искомой структуре. Таким образом решается многомерная (в случае взаимодействия двух неодинаковых молекул – 6-мерная) задача оптимизации \cite{vreven}. В результате работы алгоритма получают множество потенциально корректных моделей, которое ранжируется. Оценивают эффективность алгоритма на основании того, как высоко окажется правильная модель в полученном рейтинге.

Второй группой являются методы, основанные на белковой ко-эволюции. Эти методы изначально разрабатывались для предсказания сворачивания белка (это AlphaFold2\cite{AF2}, AlphaFold-Multimer\cite{AFM}, новый AlphaFold3\cite{AF3} и вдохновленные им алгоритмы \cite{AF_followers}) и оказались применимы и к задаче предсказания взаимодействия. Данные методы, используют информацию о родственных белках (предоставленную в явном виде или выученную), чтобы построить третичную и четвертичную структуру. И хотя, в задаче предсказания свёртки, такие алгоритмы показывают точность, сравнимую с экспериментальными методами (рентгеноструктурный анализ, ЯМР-спектроскопия и т.д.), с задачей предсказания взаимодействия они успешно справляются примерно в половине случаев, что, в свою очередь, является лучшим результатом в области.

Третьей группой обозначим методы на основе шаблонов \cite{andras}. В некотором смысле, такие алгоритмы являются классической версией алгоритмов второй группы. Этот подход так же применим к обеим задачам. Он основан на том наблюдении, что схожие участки в разных белках имеют тенденцию сворачивать/взаимодействовать сходно. Таким образом, имея базу известных свёрток/взаимодействий, поискав в ней фрагменты неизвестного соединения можно восстановить его структуру.

К последней группе отнесём различные методы глубокого обучения, предсказывающие интерфейс взаимодействия \cite{deep_methods}. Их можно разделить на несколько категорий в зависимости от того, какие архитектуры сетей используются и как представлены входные данные.
\begin{itemize}
\item Методы, использующие в качестве входных данных первичную структуру, используют свёрточные и рекуррентные сети.
\item Методы, использующие представление, в которых используется трёхмерная структура, заданная через трёхмерные воксели, используют трёхмерные свёрточные нейронные сети.
\item Методы, использующие в качестве входных данных поверхности белковых молекул \cite{hang}, используют топологические свёрточные сети \cite{sverrisson}.
\item Методы, использующие графовые представления белковых молекул, используют графовые нейронные сети.
\end{itemize}
В эту же категорию можно записать подход, при котором входными данными является представление структуры белка в виде двумерной карты расстояний \cite{guo}. При таком подходе используют полносвёрточные нейронные сети. Имея интерфейс взаимодействия, если необходимо, можно восстановить полную модель \cite{hadarovich}.

Подход, описываемый в данной работе, является развитием как раз последнего подхода. Далее, отметим несколько методов оценки качества предсказаний.
\section{Оценка качества предсказаний}

Конвенциональным методом оценки качества предсказания является метрика RMSD (root mean squared deviation) – среднеквадратическое отклонение \cite{kufareva}, измеряемое в ангстремах. Обычно для расчёта этой метрики  используют лишь C\textalpha-атомы главной цепи. Для димерных комплексов обычно фиксируют одну из компонент, и считают отклонение атомов второй (предсказанной) компоненты от их истинной позиции. Модель считают допустимой, если её RMSD меньше 10 ангстрем. Допустимые модели считаются тем качественнее, чем меньше их RMSD.

При оценке алгоритмов, предсказывающих интерфейс взаимодействия, полные модели зачастую не строят. Вместо этого оценивают сами карты интерфейсов. Предсказание этих карт, по сути, представляет собой задачу бинарной классификации, поэтому их оценивают метриками, применимыми в задачах бинарной классификации: точностью, полнотой, коэффициентом корреляции Мэтью \cite{deep_methods}, площадью под кривой ROC (AUROC) \cite{bradley}.

Существуют метрики оценки качества предсказаний, комбинирующие в себе два описанных выше метода. Такая метрика, например, используется в соревновании CAPRI \cite{basu}. Такие метрики используются потому, что нередки ситуации, когда хорошо предсказанный интерфейс порождает плохую трёхмерную модель.

\section{Ранее полученные результаты}
В предыдущих работах \cite{prip2023} был разработан подход, при котором белковые молекулы и комплексы кодировались с помощью матриц косинусов (см. \textit{Главу 2}). Это позволяло представить входные данные в независимой от сдвигов и поворотов форме, а также получать выходные данные в такой же форме. На вход подавалась информация о первичной и третичной структуре молекул, а на выходе получалась информация о повороте молекул и направлении их взаимодействия, откуда довольнно легко получалась готовая четвертичная структура. Использовались полносвёрточные нейронные сети FCN5 (см. \textit{Главу 5}). Результат был следующим:
\begin{itemize}
\item Обучающий набор данных содержал 23.344 комплексов, тестовый содержал 5.854;
\item 67\% обучающей выборки предсказывались корректно (70\% гомодимеров и 64\% гетеродимеров);
\item 51\% тестовой выборки предсказывалось корректно (59\% гомодимеров и 45\% гетеродимеров).
\end{itemize}

Дальнейший анализ (по методике \cite{zhu}) показал, что хотя численные показатели из \cite{prip2023} выглядят хорошо, общее покрытие различных протеом (семейств белков) весьма бедно, что знначит, что нейронная сеть выучила наиболее распространённые (не в природе, но в банке PDB) семейства белковых комплексов, а значит использует не столько информацию о структуре молекул, сколько выученные ко-эволюционные представления. При поступлении на вход совершенно незнакомых сети белков, сеть производит совершенно неразборчивые матрицы косинусов, то есть описаннное выше явление может быть замечено даже при визуальном анализе выхода нейронной сети.

В рамках текущей работы, основным направлением работы было построение более широкой модели белок-белкового взаимодействия. Крайне желательно, чтобы такая модель могла предсказывать белок-пептидные взаимодействия (третичная структура пептидов часто меняется при взаимодействии с большой молекулой) и учитывать белок-лигандные взаимодействия. Также желательно иметь возможность моделировать не только взаимодействие пары белков, но и комплексы более высокого порядка.






