\newpage
\begin{center}
	\textbf{\large ГЛАВА 4}

	\textbf{\large ВСПОМОГАТЕЛЬНЫЙ МАТЕМАТИЧЕСКИЙ АППАРАТ}
\end{center}
\refstepcounter{chapter}
\addcontentsline{toc}{chapter}{4. ВСПОМОГАТЕЛЬНЫЙ МАТЕМАТИЧЕСКИЙ АППАРАТ}

В рамках данной главы описаны методы, которые применялись для компьютерной реализации теоретических моделей, изложенных в предыдущей главе.

\section{Конечномерная аппроксимация поля в пространстве}
Согласно \ref{generalized_variables}, для моделирования физической системы необходимо уметь задавать некоторые обобщённые переменные в пространсте-времени (пространстве). В данной работе основной формой такого задания являются нейронные сети (по схеме \ref{generalized_variables1}). Однако, для численного поиска минимума \ref{lagrangian_principle1} необходимо много раз производить  интегрирование, и каждое интегрирование требует своего определённого набора точек. В то же время, использование нейронной сети для вычисления в каждой такой точке может быть крайне трудоёмко, поэтому вполне логично, имея нейронную сеть и некоторую белковыую молекулу, попробовать построить аппроксимацию поля, создаваемого данной молекулой, и задаваемого нейронной сетью. Использование подобной аппроксимации, хотя и само по себе трудоёмко, может в несколько раз ускорить процесс оптимизации \ref{lagrangian_principle1}. Далее описан подход, позволяющий с определённой долей успеха выполнить вышеописанную задачу. 

Для простоты начнём с одномерного случая. Пусть имеется непрерывная функция $f(x)$, $x \in \mathbb{R}$ такая, что $\exists \int_{-\infty}^{+\infty}f(x) \, dx \in \mathbb{R}$ и
$\lim_{x \to \pm\infty}f(x) = 0$. С помощью сигмоиды, переменной $x \in \mathbb{R}$ можно поставить в соответствие $t \in [0, 1]$:
\begin{equation}
	t = \frac{1}{1 + e^{-x}}.
	\label{t_and_x}
\end{equation}
Обратное преобразование выполняется с помощью функции логит:
\begin{equation}
	x = \ln\frac{t}{1-t}.
	\label{x_and_t}
\end{equation}
Зададим непрерывную функцию $g(t) = f(x) = f(\ln\frac{t}{1-t})$. Из свойств $f$ ясно, что:
\begin{equation}
	g(0) = g(1) = 0,
	\label{g_zeros}
\end{equation}
а также, что $\int_0^1g(t) \, dt \in \mathbb{R}$.

Из-за \ref{g_zeros} функция $g$ может быть продолжена как нечётная на $[-1, 1]$, а следовательно -- быть разложена в ряд Фурье по синусам:
\begin{equation}
	g(t) = \sum_{i=1}^\infty a_i\sin{\pi it},
	\label{fourier}
\end{equation}
где:
\begin{equation}
	a_i = 2\int_0^1g(t) \sin{\pi it} \, dt.
	\label{a_coef}
\end{equation}

