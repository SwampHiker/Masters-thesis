\newpage
\begin{center}
	\textbf{\large ГЛАВА 4}

	\textbf{\large ВСПОМОГАТЕЛЬНЫЙ МАТЕМАТИЧЕСКИЙ АППАРАТ}
\end{center}
\refstepcounter{chapter}
\addcontentsline{toc}{chapter}{4. ВСПОМОГАТЕЛЬНЫЙ МАТЕМАТИЧЕСКИЙ АППАРАТ}

В рамках данной главы описаны методы, которые применялись для компьютерной реализации теоретических моделей, изложенных в предыдущей главе.

\section{Конечномерная аппроксимация поля в пространстве}
Согласно \ref{generalized_variables}, для моделирования физической системы необходимо уметь задавать некоторые обобщённые переменные в пространсте-времени (пространстве). В данной работе основной формой такого задания являются нейронные сети (по схеме \ref{generalized_variables1}). Однако, для численного поиска минимума \ref{lagrangian_principle1} необходимо много раз производить  интегрирование, и каждое интегрирование требует своего определённого набора точек. В то же время, использование нейронной сети для вычисления в каждой такой точке может быть крайне трудоёмко, поэтому вполне логично, имея нейронную сеть и некоторую белковыую молекулу, попробовать построить аппроксимацию поля, создаваемого данной молекулой, и задаваемого нейронной сетью. Использование подобной аппроксимации, хотя и само по себе трудоёмко, может в несколько раз ускорить процесс оптимизации \ref{lagrangian_principle1}. Далее описан подход, позволяющий с определённой долей успеха выполнить вышеописанную задачу. 

\subsection{Определение}
Для простоты начнём с одномерного случая. Пусть имеется непрерывная функция $f(x)$, $x \in \mathbb{R}$ такая, что $\exists \intinf f(x) \, dx \in \mathbb{R}$ и
$\lim_{x \to \pm\infty}f(x) = 0$. С помощью сигмоиды, переменной $x \in \mathbb{R}$ можно поставить в соответствие $t \in [0, 1]$:
\begin{equation}
	t = \frac{1}{1 + e^{-x}}.
	\label{t_and_x}
\end{equation}
Обратное преобразование выполняется с помощью функции логит:
\begin{equation}
	x = \ln\frac{t}{1-t}.
	\label{x_and_t}
\end{equation}
Зададим непрерывную функцию $g(t) = f(x) = f(\ln\frac{t}{1-t})$. Из свойств $f$ ясно, что:
\begin{equation}
	g(0) = g(1) = 0,
	\label{g_zeros}
\end{equation}
а также, что $\int_0^1g(t) \, dt \in \mathbb{R}$.

Из-за \ref{g_zeros} функция $g$ может быть продолжена как нечётная на $[-1, 1]$, а следовательно -- быть разложена в ряд Фурье по синусам:
\begin{equation}
	g(t) = \sum_{i=1}^\infty a_i\sin{\pi it},
	\label{fourier}
\end{equation}
где:
\begin{equation}
	a_i = 2\int_0^1g(t) \sin{\pi it} \, dt.
	\label{a_coef}
\end{equation}

Ряд \ref{fourier} сходится равномерно. Через него можно записать функцию $f(x)$:
\begin{equation}
	f(x) = \sum_{i=1}^\infty a_i\sin{\frac{\pi i}{1 + e^{-x}}},
	\label{logit_fourier}
\end{equation}
а коэффициент $a_i$ перепишется как:
\begin{equation}
	a_i = 2\int_0^1f(\ln\frac{t}{1-t}) \sin{\pi it} \, dt = 2\intinf f(x)\sin(\frac{\pi i}{1+e^{-x}})\frac{e^{-x}}{(1+e^{-x})^2}\, dx.
	\label{a_coef_logit}
\end{equation}
Раз \ref{fourier} сходится равномерно, то и \ref{logit_fourier} сходится равномерно.

Далее будет необходимо воспользоваться двумя известными функциями:
\begin{itemize}
\item $\si(x) = \int_0^x\frac{\sin t}{t}\, dt$ -- интегральный синус;
\item $\cin(x) = \int_0^x\frac{1 - \cos t}{t}\, dt$ -- интегральный косинус.
\end{itemize}

Если в \ref{fourier} имеем дело с разложением функции $g(t)$ по ортонормированному базису из функций $2\sin{\pi it}$, то в случае с \ref{logit_fourier}
последовательность функций $\sin{\frac{\pi i}{1 + e^{-x}}}$ не явлется ни нормированной, ни попарно ортогональной. Рассчитаем основные интегралы, связанные с \ref{logit_fourier}:
\begin{align}
	&\intinf\sin{\frac{\pi i}{1+e^{-x}}} \, dx = \int_0^1\sin{\pi it} \, d\ln{\frac{t}{1-t}} = \int_0^1 \frac{\sin{\pi it}}{t(1-t)} \, dt = \notag \\
	&= \int_0^1 \frac{\sin{\pi it}}{t} \, dt + \int_0^1 \frac{\sin{\pi it}}{1-t} \, dt = \int_0^1 \frac{\sin{\pi it}}{t} \, dt - \int_0^1 \frac{\sin{(\pi - \pi i\xi)}}{\xi} \, d\xi = \notag \\
	&= \int_0^1 \frac{\sin{\pi it}}{t} \, dt - {(-1)}^i\int_0^1 \frac{\sin{\pi i\xi}}{\xi} \, d\xi = (1 + {(-1)}^{i+1})\si(\pi i).
	\label{f_int}
\end{align}
Формула \label{f_int} позволяет при наличии коэффициентов \ref{a_coef} посчитать интеграл функции $f(x)$ по всей числовой прямой. Пусть имеются две функции $f^1(x)$ и $f^2(x)$, которые разлагаются по \ref{logit_fourier} с коэффициентами $a^1_i$ и $a^2_j$. Крайне полезно уметь считать интеграл $\intinf f^1(x)f^2(x)\, dx$ (сравните с \ref{action_function_two} и \ref{L_partial_multiscalar}), что было бы крайне легко, если бы последовательность функций $\sin{\frac{\pi i}{1 + e^{-x}}}$ была ортонормированной, но так как это не так, необходимо рассчитать интегралы попарных произведений функций этой последовательности:

\begin{align}
	&\intinf\sin{\frac{\pi i}{1+e^{-x}}}\sin{\frac{\pi j}{1+e^{-x}}} \, dx = \int_0^1\sin{\pi it}\sin{\pi jt} \, d\ln{\frac{t}{1-t}} = \notag \\
	&= \int_0^1\frac{\cos\pi (i-j)t - \cos\pi (i+j)t}{2}\left(\frac{1}{t} + \frac{1}{1-t}\right)\, dt = \frac{1}{2}(\int_0^1\frac{1-\cos\pi (i+j)t}{t}\, dt - \notag \\
	&- \int_0^1\frac{1-\cos\pi (i-j)t}{t}\, dt + \int_0^1\frac{\cos\pi (i-j)t-\cos\pi (i+j)t}{1-t}\, dt) = \frac{1}{2}(\cin(\pi(i+j)) -\notag \\
	&- \cin(\pi(i-j)) + \int_0^1\frac{{(-1)^{i-j}}\cos\pi(i-j)\xi - {(-1)^{i+j}}\cos\pi(i+j)\xi}{\xi}\, d\xi) = \notag \\
	&= \frac{1+{(-1)}^{i+j}}{2}\left(\cin(\pi(i+j)) - \cin(\pi(i-j))\right).
	\label{fg_int}
\end{align}

Из \ref{fg_int} при $i=j$ получим:
\begin{equation}
	\intinf\sin^2\frac{\pi i}{1+e^{-x}}\, dx = \cin(2\pi i).
	\label{f_sqr_int}
\end{equation}

Так как в рамках исследуемой задачи, необходимо аппроксимировать функции заданные на трёхмерном, а не одномерном пространстве, то следует перейти к трёхмерному ряду Фурье.
Для такого ряда формулу \ref{logit_fourier} перепишем как:
\begin{equation}
	f(x,y,z) = \sum_{i,j,k=1}^\infty A_{i,j,k}\sin{\frac{\pi i}{1 + e^{-x}}}\sin{\frac{\pi j}{1 + e^{-y}}}\sin{\frac{\pi k}{1 + e^{-z}}},
	\label{logit_fourier_3d}
\end{equation}
где:
\begin{equation}
	A_{i,j,k} = 2^3\iiint_{\mathbb{R}^3}\frac{f(x, y, z)\sin\frac{\pi i}{(1+e^{-x})^2}\sin\frac{\pi j}{(1+e^{-y})^2}\sin\frac{\pi k}{(1+e^{-z})^2}e^{-x-y-z}}{(1+e^{-x})^2(1+e^{-y})^2(1+e^{-z})^2}\, d^3x.
	\label{a_coef_logit_3d}
\end{equation}

Легко видеть, что формулы \ref{f_int}, \ref{fg_int} и \ref{f_sqr_int} полезны и для трёхмерного случая.

\subsection{Вычисление коэффициентов по методу Монте-Карло}

При построении аппроксимации, во-первых, приходится ограничиваться некоторым конечным числом слагаемых ряда, а во-вторых, необходимо численно получать значения коэффициентов \ref{a_coef_logit_3d}. Из-за относительно большой размерности функции, и необходимости считать интегралы по всему пространству, интегрирование по сетке точек не так удобно -- проще применить метод Монте-Карло.

Для метода Монте-Карло необходимо иметь некоторую трёхмерную случайную величину $\xi = {[\xi_x, \xi_y, \xi_z]}^\mathrm{T}$, принимающую все значения из $\mathbb{R}^3$ с распределением вероятности $p(x, y, z)$ ($p > 0$, $\iiint_{\mathbb{R}^3}p(x, y, z)\, dx\, dy\, dz = 1$). На роль такой случайной величины хорошо подходит многомерное нормальное распределение. Хорошо известно, что для любой функции $f(x, y, z)$ можно записать:
\begin{equation}
	\iiint_{\mathbb{R}^3}f(x,y,z)\, dx\, dy\, dz=\iiint_{\mathbb{R}^3}\frac{f(x,y,z)}{p(x,y,z)}p(x,y,z)\, dx\, dy\, dz=E\{\frac{f(\xi)}{p(\xi)}\},
	\label{monte-carlo1}
\end{equation}
из чего, при наличии $N$ выборочных значений $\xi^i$, через статистическую оценку среднего получаем:
\begin{equation}
	\iiint_{\mathbb{R}^3}f(x,y,z)\, dx\, dy\, dz=E\{\frac{f(\xi)}{p(\xi)}\} \approx \frac{1}{N}\sum_{i=1}^N\frac{f(\xi^i)}{p(\xi^i)}.
	\label{monte-carlo2}
\end{equation}
Использование формулы \ref{monte-carlo2} совместно с \ref{a_coef_logit_3d} даёт искомую аппроксимацию.

Достоинством описанного выше метода является, как минимум, то, что он работает, и позволяет с некоторой точностью записать многомерную, заданную на всём бесконечном множестве функцию, используя конечное число параметров. Следует также привести недостатки данного метода:
\begin{enumerate}
\item При ограниченном числе членов ряда, точность приближения крайне чувствительна к сдвигам приближаемой функции ($f(x, y, z) \to f(x+c_x, y+c_y, z+c_z)$). Это выражается в том плане, что  не следует сразу использовать \ref{a_coef_logit_3d}, вместо этого следует искать такое преобразование исходных координат -- желательно линейное $x = A\hat{x} + b$, для которого отклонение аппроксимации $\hat{f}$ от исходной $f$ будет минимальным:
\begin{align}
	&A'_{i,j,k} = 2^3\iiint_{\mathbb{R}^3}\frac{f(Ax+b)\sin\frac{\pi i}{(1+e^{-(a_{11}x+a_{12}y+a_{13}z+b_1)})^2}\cdot...\cdot e^{-((a_{11}+...)x+...)}}{(1+e^{-(a_{11}x+a_{12}y+a_{13}z+b_1)})^2...}\, d^3x,
	\label{a_coef_logit_3d_mod}
\end{align}
\begin{equation}
	\hat{f}(x,y,z) = \det(A)\sum_{i,j,k=1}^\infty A'_{i,j,k}\sin{\frac{\pi i}{1 + e^{-(a_{11}x+a_{12}y+a_{13}z+b_1)}}}\cdot...,
	\label{logit_fourier_3d_mod}
\end{equation}
и
\begin{equation}
	\iiint_{{[0, 1]}^3}{(f(x) - \hat{f}(x))}^2\,d^3x \to \min.
	\label{a_coef_logit_3d_mod_cond}
\end{equation}
Поиск \ref{a_coef_logit_3d_mod_cond} эффективно выполняется с помощью методов градиентного спуска.

\item Точность приближения также сильно зависит от дисперсии распределения $\xi$. Если дисперсия слишком мала, она может не покрыть всех особенностей функции $f$.

\item Увеличение количества учитываемых членов ряда Фурье приводит к увеличению точности, но в многомерном случае приводит к очень быстрому росту трудоёмкости вычислений, а формулы \ref{f_int}, \ref{fg_int} и \ref{f_sqr_int} становятся малоприменимыми.

\item Метод Монте-Карло обладает высокой дисперсией, его точность растёт со скоростью $O(\frac{1}{\sqrt{N}})$. Вместе с разложением в ряд Фурье и недостаточно большими значениями $N$ приводят к бвстрому накоплению ошибок.
\end{enumerate}

\subsection{Графические примеры}
% Тут подгрузить картинок
картинки:)

\section{Параметризация пары взаимодействующих молекул}
todo

\section{Алгоритмы оптимизации}
todo
