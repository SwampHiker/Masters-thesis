\documentclass[
candidate, % document type
subf, % use and configure subfig package for nested figure numbering
times new roman % use Times New Roman font as main
]{disser}

% Кодировка и язык
\usepackage[T2A]{fontenc} % поддержка кириллицы
\usepackage[utf8]{inputenc} % кодировка исходного текста
\usepackage[english,russian]{babel} % переключение языков
\usepackage{textgreek} % греческие букву в текстовом режиме

% Геометрия страницы и графика
\usepackage[a4paper, left=3cm, right=15mm, top=2cm, bottom=2cm]{geometry} % поля страницы
\usepackage{graphicx} % подключение графики
\usepackage{pdfpages} % вставка pdf-страниц

% Таблицы
\usepackage{array} % расширенные возможности для работы с таблицами
\usepackage{tabularx} % автоматический подбор ширины столбцов
\usepackage{dcolumn} % выравнивание чисел по разделителю

% Математика
\usepackage{bm} % полужирное начертание для математических символов
\usepackage{amsmath} % дополнительные математические возможности
\usepackage{amssymb} % дополнительные математические символы
\usepackage{algorithmic} % описание алгоритмов

% Библиография и ссылки
\usepackage{cite} % поддержка цитирования
%\usepackage[hidelinks]{hyperref} % создание гиперссылок
\let\oldref\ref
\renewcommand{\ref}[1]{(\oldref{#1})} % скобки вокруг ссылок

% Прочее
\usepackage{color} % работа с цветом
\usepackage{epstopdf} % конвертация eps в pdf
\usepackage{multirow} % объединение ячеек таблиц по вертикали
\usepackage{afterpage} % вставка материала после текущей страницы
\usepackage[font={normal}]{caption} % настройка подписей к рисункам и таблицам
\usepackage[singlespacing]{setspace} % один межстрочный интервал
\usepackage{fancyhdr} % установка колонтитулов
\usepackage{listings} % поддержка вставки исходного кода

% Установка шрифта Times New Roman
\usepackage{tempora}

% Создание нового типа столбца для выравнивания содержимого по центру
\newcommand{\PreserveBackslash}[1]{\let\temp=\\#1\let\\=\temp}
\newcolumntype{C}[1]{>{\PreserveBackslash\centering}p{#1}}

% Настройка стиля страницы
\pagestyle{fancy}      % Использование стиля "fancy" для оформления страниц
\fancyhf{}              % Очистка текущих значений колонтитулов
\fancyfoot[C]{\thepage} % Установка номера страницы в нижнем колонтитуле по центру
\renewcommand{\headrulewidth}{0pt} % Удаление разделительной линии в верхнем колонтитуле
\setlength{\parindent}{1.25cm} % Установка абзацного отступа

% Настройка подписей к изображениям и таблицам
\captionsetup{format=hang,labelsep=period}

% Установка глубины оглавления
\setcounter{tocdepth}{2}

% Указание папки для поиска изображений
\graphicspath{{images/}}

% Установка стилей страницы и главы
\pagestyle{footcenter}
\chapterpagestyle{footcenter}


%сдвиг в списк
\usepackage{enumitem}
\setlist{leftmargin=12.5mm}

%дополнительные математические операторы
\DeclareMathOperator{\diag}{diag}
\DeclareMathOperator{\Sqrt}{sqrt}
\DeclareMathOperator{\const}{const}
\DeclareMathOperator{\argmin}{argmin}
\DeclareMathOperator{\tr}{tr}
\newcommand{\intinf}{\int_{-\infty}^{+\infty}}
\DeclareMathOperator{\si}{Si}
\DeclareMathOperator{\cin}{Cin}

% метаданные PDF
%\hypersetup{
%	pdfauthor={Новиков А.А.},
%	pdftitle={Алгоритмы глубокого обучения для предсказания структуры белковых комплексов},
%	pdfkeywords={},
%	pdfsubject={},
%	pdfcreator={pdflatex},
%	pdflang={Russian}}