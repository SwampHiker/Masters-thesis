%\setcounter{page}{3}
\newpage
\renewcommand{\contentsname}{\centerline{\large ОГЛАВЛЕНИЕ}}
\tableofcontents

\newpage
\addcontentsline{toc}{chapter}{ВВЕДЕНИЕ}
%\onehalfspacing
\begin{center}
	\textbf{\large ВВЕДЕНИЕ}
\end{center}

Задача предсказания взаимодействия белков популярна среди биоинформатиков и специалистов в области структурной биологии уже на протяжении нескольких десятков лет. В последнее десятилетие, вместе с ростом актуальности нейронных сетей, были предложены новые подходы для решения этой задачи \cite{deep_methods}.

В предыдущих работах \cite{prip2023} было предложено новое представление для белковых молекул -- матрица косинусов. Данное представление использовалось в качестве входных и выходных данных полносвёрточной нейронной сети \cite{fully_conv}, предсказывавашей взаимодействие белок-белковой пары.
В рамках данной работы предпринята попытка далее развить данное представление, чтобы получить возможность применить формализм теории поля для описания белковых взаимодействий, после чего применить подходы глубокого машинного обучения, для нахождения соответствующих функций поля.
Таким образом, направление работы, в определённой степени, вернулось к классическому методу потенциалов.