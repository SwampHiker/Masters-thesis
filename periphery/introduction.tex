%\setcounter{page}{3}
\newpage
\renewcommand{\contentsname}{\centerline{\large ОГЛАВЛЕНИЕ}}
\tableofcontents

\newpage
\addcontentsline{toc}{chapter}{ВВЕДЕНИЕ}
%\onehalfspacing
\begin{center}
	\textbf{\large ВВЕДЕНИЕ}
\end{center}

Задача предсказания взаимодействия белков является одной из нерешённых задач биоинформатики и структурной биологии и остаётся актуальной на протяжении уже нескольких десятков лет. В настоящее время большой прогресс в решении этой задачи был достигнут благодаря использованию нейронных сетей глубокого обучения.

В рамках предыдущих работ \cite{prip2023} был предложен оригинальный подход к предсказанию взаимодействия, с использованием полносвёрточных нейронных сетей \cite{fully_conv} и, так называемых, матриц косинусов. Впрочем, указанный подход был ограничен предсказанием взаимодействия лишь пар больших белковых молекул. Основным направлением данной работы была попытка обобщить и расширить данный подход через создание общей модели взаимодействия биомолекул. Такая модель должна основываться на принципах разработанного ранее предсказательного подхода, но быть достаточно свободной, и позволять моделировать, в том числе, одновременное взаимодействие нескольких больших и малых белковых молекул.


% Задача предсказания взаимодействия белков популярна среди биоинформатиков и специалистов в области структурной биологии уже на протяжении нескольких десятков лет. В последнее десятилетие, вместе с ростом актуальности нейронных сетей, были предложены новые подходы для решения этой задачи \cite{deep_methods}.

%В предыдущих работах \cite{prip2023} было предложено новое представление для белковых молекул -- матрица косинусов. Данное представление использовалось в качестве входных и выходных данных полносвёрточной нейронной сети \cite{fully_conv}, предсказывавашей взаимодействие белок-белковой пары.
%В рамках данной работы предпринята попытка далее развить данное представление, чтобы получить возможность применить формализм теории поля для описания белковых взаимодействий, после чего применить подходы глубокого машинного обучения, для нахождения соответствующих функций поля.
%Таким образом, направление работы, в определённой степени, вернулось к классическому методу потенциалов.