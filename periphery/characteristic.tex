% RUSSIAN
\newpage
\addcontentsline{toc}{chapter}{ОБЩАЯ ХАРАКТЕРИСТИКА РАБОТЫ}

\begin{center}
	\textbf{\large ОБЩАЯ ХАРАКТЕРИСТИКА РАБОТЫ}
\end{center}


\textbf{Ключевые слова:} БЕЛОК, ВЗАИМОДЕЙСТВИЕ БЕЛКОВ, ГЛУБОКОЕ ОБУЧЕНИЕ, ПОЛНОСВЁРТОЧНЫЕ СЕТИ, МАТРИЦЫ КОСИНУСОВ, СВОРАЧИВАНИЕ БЕЛКА, ТЕОРИЯ ПОЛЯ, СПЕЦИАЛЬНАЯ ТЕОРИЯ ОТНОСИТЕЛЬНОСТИ.

\textbf{Цель работы:} предсказание взаимодействия пар белковых молекул, заданных с помощью координат C\textalpha-атомов углерода и последовательностей аминокислотных остатков.

\textbf{Объект исследования:} белковые молекулы и комплексы.

\textbf{Предмет исследования:} белок-белковое взаимодействие, способы его описания и методы предсказания.

\textbf{Методы исследования:} методы математического анализа и линейной алгебры, методы глубокого машинного обучения.

\textbf{Результаты работы:} получены новые теоретические результаты, проведены и проанализированы вычислительные эксперименты.

\textbf{Области применения:} биоинформатика, медицина, физика, математическое моделирование.

\textbf{Структура магистерской диссертации:} работа изложена на 58 страницах, состоит из общей характеристики на 3 языках, введения, 6 глав, заключения и списка использованных источников. Содержит 18 рисунков.

% BELARUSIAN
\newpage
\addcontentsline{toc}{chapter}{АГУЛЬНАЯ ХАРАКТЫРЫСТЫКА РАБОТЫ}
\begin{center}
	\textbf{\large АГУЛЬНАЯ ХАРАКТЫРЫСТЫКА РАБОТЫ}
\end{center}

\textbf{Ключавыя словы:} БЯЛОК, УЗАЕМАДЗЕЯННЕ БЯЛКОЎ, ГЛЫБОКАЕ НАВУЧАННЕ, ПОЎНАЗГОРТАЧНЫЯ СЕТКІ, МАТРЫЦЫ КОСІНУСАЎ, ЗГОРТВАННЕ БЯЛКУ, ТЭОРЫЯ ПОЛЯ, СПЕЦЫЯЛЬНАЯ ТЭОРЫЯ АДНОСНАСЦІ.

\textbf{Мэта работы:} прадказанне ўзаемадзеяння пар бялковых малекул, зададзеных з дапамогай каардынат C\textalpha-атамаў вугляроду і паслядоўнасцяў амінакіслотных астаткаў.

\textbf{Аб’ектам даследавання:} бялковыя малекулы і комплексы.

\textbf{Прадмет даследавання:} бялок-бялковае ўзаемадзеянне, спосабы яго апісання і метады прадказання.

\textbf{Метады даследавання:} метады матэматычнага аналізу і лінейнай алгебры, метады глыбокага машыннага навучання.

\textbf{Вынікі работы:} атрыманы новыя тэарэтычныя вынікі, праведзены і прааналізаваны вылічальныя эксперыменты.

\textbf{Вобласть ўжывання:} біяінфарматыка, медыцына, фізіка, матэматычнае мадэляванне.

\textbf{Структура магістарскай дысертацыі:} праца выкладзена на 58 старонках, складаецца з агульнай характарыстыкі на 3 мовах, увядзення, 6 глаў, заключэння і спісу выкарыстаных крыніц. Змяшчае 18 малюнкаў.

% ENGLISH
\newpage
\addcontentsline{toc}{chapter}{GENERAL DESCIPTION OF WORK}
\begin{center}
	\textbf{\large GENERAL DESCRIPTION OF WORK}
\end{center}

\textbf{Keywords:} PROTEIN, PROTEIN INTERACTION, DEEP LEARNING, FULLY CONVOLUTIONAL NETWORKS, COSINE MATRICES, PROTEIN FOLDING, EIGENVECTORS, FIELD THEORY, SPECIAL RELATIVITY.

\textbf{The aim:} interaction prediction for pairs of protein molecules specified with C\textalpha-carbon atoms coordinates and sequences of amino acid residues.

\textbf{The object:} protein molecules and complexes.

\textbf{Research methods:} methods of calculus and linear algebra, deep machine learning methods.

\textbf{The results:} new theoretical results were obtained, computational experiments were conducted and analyzed.

\textbf{Application:} bioinformatics, medicine, physics, mathematical modelling.

\textbf{Structure of the Master's Thesis:} the work is presented on 58 pages, consists of a general desciption in 3 languages, an introduction, 6 chapters, a conclusion and a list of references. It contains 18 figures.
